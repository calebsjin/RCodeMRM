% Options for packages loaded elsewhere
\PassOptionsToPackage{unicode}{hyperref}
\PassOptionsToPackage{hyphens}{url}
%
\documentclass[
  letterpaper,
]{book}

\usepackage{amsmath,amssymb}
\usepackage{lmodern}
\usepackage{iftex}
\ifPDFTeX
  \usepackage[T1]{fontenc}
  \usepackage[utf8]{inputenc}
  \usepackage{textcomp} % provide euro and other symbols
\else % if luatex or xetex
  \usepackage{unicode-math}
  \defaultfontfeatures{Scale=MatchLowercase}
  \defaultfontfeatures[\rmfamily]{Ligatures=TeX,Scale=1}
\fi
% Use upquote if available, for straight quotes in verbatim environments
\IfFileExists{upquote.sty}{\usepackage{upquote}}{}
\IfFileExists{microtype.sty}{% use microtype if available
  \usepackage[]{microtype}
  \UseMicrotypeSet[protrusion]{basicmath} % disable protrusion for tt fonts
}{}
\makeatletter
\@ifundefined{KOMAClassName}{% if non-KOMA class
  \IfFileExists{parskip.sty}{%
    \usepackage{parskip}
  }{% else
    \setlength{\parindent}{0pt}
    \setlength{\parskip}{6pt plus 2pt minus 1pt}}
}{% if KOMA class
  \KOMAoptions{parskip=half}}
\makeatother
\usepackage{xcolor}
\setlength{\emergencystretch}{3em} % prevent overfull lines
\setcounter{secnumdepth}{5}
% Make \paragraph and \subparagraph free-standing
\ifx\paragraph\undefined\else
  \let\oldparagraph\paragraph
  \renewcommand{\paragraph}[1]{\oldparagraph{#1}\mbox{}}
\fi
\ifx\subparagraph\undefined\else
  \let\oldsubparagraph\subparagraph
  \renewcommand{\subparagraph}[1]{\oldsubparagraph{#1}\mbox{}}
\fi

\usepackage{color}
\usepackage{fancyvrb}
\newcommand{\VerbBar}{|}
\newcommand{\VERB}{\Verb[commandchars=\\\{\}]}
\DefineVerbatimEnvironment{Highlighting}{Verbatim}{commandchars=\\\{\}}
% Add ',fontsize=\small' for more characters per line
\usepackage{framed}
\definecolor{shadecolor}{RGB}{241,243,245}
\newenvironment{Shaded}{\begin{snugshade}}{\end{snugshade}}
\newcommand{\AlertTok}[1]{\textcolor[rgb]{0.68,0.00,0.00}{#1}}
\newcommand{\AnnotationTok}[1]{\textcolor[rgb]{0.37,0.37,0.37}{#1}}
\newcommand{\AttributeTok}[1]{\textcolor[rgb]{0.40,0.45,0.13}{#1}}
\newcommand{\BaseNTok}[1]{\textcolor[rgb]{0.68,0.00,0.00}{#1}}
\newcommand{\BuiltInTok}[1]{\textcolor[rgb]{0.00,0.23,0.31}{#1}}
\newcommand{\CharTok}[1]{\textcolor[rgb]{0.13,0.47,0.30}{#1}}
\newcommand{\CommentTok}[1]{\textcolor[rgb]{0.37,0.37,0.37}{#1}}
\newcommand{\CommentVarTok}[1]{\textcolor[rgb]{0.37,0.37,0.37}{\textit{#1}}}
\newcommand{\ConstantTok}[1]{\textcolor[rgb]{0.56,0.35,0.01}{#1}}
\newcommand{\ControlFlowTok}[1]{\textcolor[rgb]{0.00,0.23,0.31}{#1}}
\newcommand{\DataTypeTok}[1]{\textcolor[rgb]{0.68,0.00,0.00}{#1}}
\newcommand{\DecValTok}[1]{\textcolor[rgb]{0.68,0.00,0.00}{#1}}
\newcommand{\DocumentationTok}[1]{\textcolor[rgb]{0.37,0.37,0.37}{\textit{#1}}}
\newcommand{\ErrorTok}[1]{\textcolor[rgb]{0.68,0.00,0.00}{#1}}
\newcommand{\ExtensionTok}[1]{\textcolor[rgb]{0.00,0.23,0.31}{#1}}
\newcommand{\FloatTok}[1]{\textcolor[rgb]{0.68,0.00,0.00}{#1}}
\newcommand{\FunctionTok}[1]{\textcolor[rgb]{0.28,0.35,0.67}{#1}}
\newcommand{\ImportTok}[1]{\textcolor[rgb]{0.00,0.46,0.62}{#1}}
\newcommand{\InformationTok}[1]{\textcolor[rgb]{0.37,0.37,0.37}{#1}}
\newcommand{\KeywordTok}[1]{\textcolor[rgb]{0.00,0.23,0.31}{#1}}
\newcommand{\NormalTok}[1]{\textcolor[rgb]{0.00,0.23,0.31}{#1}}
\newcommand{\OperatorTok}[1]{\textcolor[rgb]{0.37,0.37,0.37}{#1}}
\newcommand{\OtherTok}[1]{\textcolor[rgb]{0.00,0.23,0.31}{#1}}
\newcommand{\PreprocessorTok}[1]{\textcolor[rgb]{0.68,0.00,0.00}{#1}}
\newcommand{\RegionMarkerTok}[1]{\textcolor[rgb]{0.00,0.23,0.31}{#1}}
\newcommand{\SpecialCharTok}[1]{\textcolor[rgb]{0.37,0.37,0.37}{#1}}
\newcommand{\SpecialStringTok}[1]{\textcolor[rgb]{0.13,0.47,0.30}{#1}}
\newcommand{\StringTok}[1]{\textcolor[rgb]{0.13,0.47,0.30}{#1}}
\newcommand{\VariableTok}[1]{\textcolor[rgb]{0.07,0.07,0.07}{#1}}
\newcommand{\VerbatimStringTok}[1]{\textcolor[rgb]{0.13,0.47,0.30}{#1}}
\newcommand{\WarningTok}[1]{\textcolor[rgb]{0.37,0.37,0.37}{\textit{#1}}}

\providecommand{\tightlist}{%
  \setlength{\itemsep}{0pt}\setlength{\parskip}{0pt}}\usepackage{longtable,booktabs,array}
\usepackage{calc} % for calculating minipage widths
% Correct order of tables after \paragraph or \subparagraph
\usepackage{etoolbox}
\makeatletter
\patchcmd\longtable{\par}{\if@noskipsec\mbox{}\fi\par}{}{}
\makeatother
% Allow footnotes in longtable head/foot
\IfFileExists{footnotehyper.sty}{\usepackage{footnotehyper}}{\usepackage{footnote}}
\makesavenoteenv{longtable}
\usepackage{graphicx}
\makeatletter
\def\maxwidth{\ifdim\Gin@nat@width>\linewidth\linewidth\else\Gin@nat@width\fi}
\def\maxheight{\ifdim\Gin@nat@height>\textheight\textheight\else\Gin@nat@height\fi}
\makeatother
% Scale images if necessary, so that they will not overflow the page
% margins by default, and it is still possible to overwrite the defaults
% using explicit options in \includegraphics[width, height, ...]{}
\setkeys{Gin}{width=\maxwidth,height=\maxheight,keepaspectratio}
% Set default figure placement to htbp
\makeatletter
\def\fps@figure{htbp}
\makeatother

\makeatletter
\makeatother
\makeatletter
\@ifpackageloaded{bookmark}{}{\usepackage{bookmark}}
\makeatother
\makeatletter
\@ifpackageloaded{caption}{}{\usepackage{caption}}
\AtBeginDocument{%
\ifdefined\contentsname
  \renewcommand*\contentsname{Table of contents}
\else
  \newcommand\contentsname{Table of contents}
\fi
\ifdefined\listfigurename
  \renewcommand*\listfigurename{List of Figures}
\else
  \newcommand\listfigurename{List of Figures}
\fi
\ifdefined\listtablename
  \renewcommand*\listtablename{List of Tables}
\else
  \newcommand\listtablename{List of Tables}
\fi
\ifdefined\figurename
  \renewcommand*\figurename{Figure}
\else
  \newcommand\figurename{Figure}
\fi
\ifdefined\tablename
  \renewcommand*\tablename{Table}
\else
  \newcommand\tablename{Table}
\fi
}
\@ifpackageloaded{float}{}{\usepackage{float}}
\floatstyle{ruled}
\@ifundefined{c@chapter}{\newfloat{codelisting}{h}{lop}}{\newfloat{codelisting}{h}{lop}[chapter]}
\floatname{codelisting}{Listing}
\newcommand*\listoflistings{\listof{codelisting}{List of Listings}}
\makeatother
\makeatletter
\@ifpackageloaded{caption}{}{\usepackage{caption}}
\@ifpackageloaded{subcaption}{}{\usepackage{subcaption}}
\makeatother
\makeatletter
\@ifpackageloaded{tcolorbox}{}{\usepackage[many]{tcolorbox}}
\makeatother
\makeatletter
\@ifundefined{shadecolor}{\definecolor{shadecolor}{rgb}{.97, .97, .97}}
\makeatother
\makeatletter
\makeatother
\ifLuaTeX
  \usepackage{selnolig}  % disable illegal ligatures
\fi
\IfFileExists{bookmark.sty}{\usepackage{bookmark}}{\usepackage{hyperref}}
\IfFileExists{xurl.sty}{\usepackage{xurl}}{} % add URL line breaks if available
\urlstyle{same} % disable monospaced font for URLs
\hypersetup{
  pdftitle={R code example for fast computing algorithm},
  pdfauthor={Shiqiang Jin and Gyuhyeong Goh},
  hidelinks,
  pdfcreator={LaTeX via pandoc}}

\title{R code example for fast computing algorithm}
\author{Shiqiang Jin and Gyuhyeong Goh}
\date{11/14/22}

\begin{document}
\frontmatter
\maketitle
\ifdefined\Shaded\renewenvironment{Shaded}{\begin{tcolorbox}[frame hidden, enhanced, borderline west={3pt}{0pt}{shadecolor}, breakable, sharp corners, boxrule=0pt, interior hidden]}{\end{tcolorbox}}\fi

\renewcommand*\contentsname{Table of contents}
{
\setcounter{tocdepth}{2}
\tableofcontents
}
\mainmatter
\bookmarksetup{startatroot}

\hypertarget{r-code-example-for-fast-computing-algorithm}{%
\chapter{R code example for fast computing
algorithm}\label{r-code-example-for-fast-computing-algorithm}}

\newcommand{\uA}{{\bf A}}
\newcommand{\ua}{{\bf a}}
\newcommand{\uB}{{\bf B}}
\newcommand{\ub}{{\bf b}}
\newcommand{\uC}{{\bf C}}
\newcommand{\uc}{{\bf c}}
\newcommand{\ud}{{\bf d}}
\newcommand{\ue}{{\bf e}}
\newcommand{\uE}{{\bf E}}
\newcommand{\uH}{{\bf H}}
\newcommand{\uI}{{\bf I}}
\newcommand{\uK}{{\bf K}}
\newcommand{\bP}{{\bf P}}
\newcommand{\uQ}{{\bf Q}}
\newcommand{\uv}{{\bf v}}
\newcommand{\uV}{{\bf V}}
\newcommand{\us}{{\bf s}}
\newcommand{\T}{{ \mathrm{\scriptscriptstyle T} }}
\newcommand{\uU}{{\bf U}}
\newcommand{\uu}{{\bf u}}
\newcommand{\uX}{{\bf X}}
\newcommand{\ux}{{\bf x}}
\newcommand{\uY}{{\bf Y}}
\newcommand{\uy}{{\bf y}}

\newcommand{\0}{{\bf 0}}
\newcommand{\1}{{\boldsymbol 1}}
\newcommand{\ualpha}{{\boldsymbol \alpha}}
\newcommand{\ubeta}{{\boldsymbol \beta}}
\newcommand{\diag}{{\rm diag}}
\newcommand{\uepsilon}{{\boldsymbol \epsilon}}
\newcommand{\ueta}{{\boldsymbol \eta}}
\newcommand{\bg}{{\boldsymbol \gamma}}
\newcommand{\bOmega}{{\boldsymbol\Omega}}
\newcommand{\uPsi}{{\boldsymbol \Psi}}
\newcommand{\uSigma}{{\boldsymbol \Sigma}}
\newcommand{\uxi}{{\boldsymbol \xi}}
\newcommand{\nbd}{{\text{nbd}}}

This is a supplementary document of a couple of R code examples
displaying the time costs using for-loop method compared with fast
computing algorithm in the paper \textbf{Bayesian best subset selection
by hybrid search under multivariate regression model of high dimensional
data}.

The R code examples compute marginal likelihood functions only in
addition neighbor of the current model as the code for deletion neighbor
is similar. The running time of executing fast computing algorithm and
for-loop method is recorded to demonstrate the high efficiency of the
proposed method.

\bookmarksetup{startatroot}

\hypertarget{review-of-marginal-likelihoods-in-addition-neighbor}{%
\chapter{Review of marginal likelihoods in addition
neighbor}\label{review-of-marginal-likelihoods-in-addition-neighbor}}

For any \(i\notin\hat{\boldsymbol \gamma}\),
\(|\hat{\boldsymbol \gamma}\cup\{i\}|=k+1\), hence
\(s({\bf Y}|\hat{\boldsymbol \gamma}\cup\{i\})\) in Eq.(6) of the paper
can be expressed as

\begin{equation}\protect\hypertarget{eq-1}{}{
s({\bf Y}|\hat{\boldsymbol \gamma}\cup\{i\})={\zeta}^{-\frac{m(k+1)}{2}}|{\bf X}_{\hat{\boldsymbol \gamma}\cup\{i\}}^{{ \mathrm{\scriptscriptstyle T} }}{\bf X}_{\hat{\boldsymbol \gamma}\cup\{i\}}+\zeta^{-1}{\bf I}_{k+1}|^{-\frac{m}{2}}| {\bf Y}^{{ \mathrm{\scriptscriptstyle T} }}{\bf H}_{\hat{\boldsymbol \gamma}\cup\{i\}}{\bf Y}+{\boldsymbol \Psi}|^{-\frac{n+\nu}{2}}.
}\label{eq-1}\end{equation}

With fast computing algorithm, we have
\begin{equation}\protect\hypertarget{eq-2}{}{\begin{eqnarray}
{\bf s}_+(\hat{\boldsymbol \gamma}) &=& c_{\hat{\boldsymbol \gamma}}^+\times \left(\zeta^{-1}{\boldsymbol 1}_p+{\rm diag}({\bf X}^{{ \mathrm{\scriptscriptstyle T} }}{\bf H}_{\hat{{\boldsymbol \gamma}}}{\bf X})\right)^{-\frac{m}{2}}\boldsymbol{\cdot}\\
&&\left[{\boldsymbol 1}_p - \frac{{\rm diag}({\bf X}^{{ \mathrm{\scriptscriptstyle T} }}{\bf H}_{\hat{{\boldsymbol \gamma}}}{\bf Y}({\bf Y}^{{ \mathrm{\scriptscriptstyle T} }}{\bf H}_{\hat{{\boldsymbol \gamma}}}{\bf Y}+{\boldsymbol \Psi})^{-1}{\bf Y}^{{ \mathrm{\scriptscriptstyle T} }}{\bf H}_{\hat{{\boldsymbol \gamma}}}{\bf X})}{\zeta^{-1}{\boldsymbol 1}_p+{\rm diag}({\bf X}^{{ \mathrm{\scriptscriptstyle T} }}{\bf H}_{\hat{{\boldsymbol \gamma}}}{\bf X})}\right]^{-\frac{n+\nu}{2}}
\end{eqnarray}}\label{eq-2}\end{equation} where
\({\bf a}^x = (a_1^x,\ldots,a_p^x)\),
\({\bf a}\boldsymbol{\cdot}{\bf b}= (a_1b_1,\ldots,a_pb_p)\),
\({\bf a}/{\bf b}= (a_1/b_1,\ldots,a_p/b_p)\) for generic vectors
\({\bf a}\) and \({\bf b}\), and
\(c_{\hat{\boldsymbol \gamma}}^+= {\zeta}^{-\frac{m(k+1)}{2}}|{\bf X}_{\hat{\boldsymbol \gamma}}^{{ \mathrm{\scriptscriptstyle T} }}{\bf X}_{\hat{\boldsymbol \gamma}}+\zeta^{-1}{\bf I}_{k}|^{-\frac{m}{2}}\left|{\bf Y}^{{ \mathrm{\scriptscriptstyle T} }}{\bf H}_{\hat{{\boldsymbol \gamma}}}{\bf Y}+{\boldsymbol \Psi}\right|^{-\frac{n+\nu}{2}}\)
is a constant with respect to \(i\notin \hat{\boldsymbol \gamma}\).

Take logarithm of Equation~\ref{eq-2}, we have
\begin{equation}\protect\hypertarget{eq-3}{}{\begin{eqnarray}
\log({\bf s}_+(\hat{\boldsymbol \gamma})) = \log(c_{\hat{\boldsymbol \gamma}}^+){\boldsymbol 1}_p-\frac{m}{2}\log({\bf d})-\frac{n+\nu}{2}\log(1-\frac{{\bf u}}{{\bf d}}),
\end{eqnarray}}\label{eq-3}\end{equation} where
\({\bf d}= \zeta^{-1}{\boldsymbol 1}_p+{\rm diag}({\bf X}^{{ \mathrm{\scriptscriptstyle T} }}{\bf H}_{\hat{{\boldsymbol \gamma}}}{\bf X})\)
and
\({\bf u}={\rm diag}({\bf X}^{{ \mathrm{\scriptscriptstyle T} }}{\bf H}_{\hat{{\boldsymbol \gamma}}}{\bf Y}({\bf Y}^{{ \mathrm{\scriptscriptstyle T} }}{\bf H}_{\hat{{\boldsymbol \gamma}}}{\bf Y}+{\boldsymbol \Psi})^{-1}{\bf Y}^{{ \mathrm{\scriptscriptstyle T} }}{\bf H}_{\hat{{\boldsymbol \gamma}}}{\bf X})\).

In the following R code example, we will evaluate
\({\bf s}({\bf Y}|{\text{nbd}}_+(\hat{\boldsymbol \gamma}))\) by
Equation~\ref{eq-1} with the for-loop method and by Equation~\ref{eq-3}
with fast computing algorithm, respectively.

\bookmarksetup{startatroot}

\hypertarget{r-code-example-of-eq-1-and-eq-3}{%
\chapter{\texorpdfstring{R code example of Equation~\ref{eq-1} and
Equation~\ref{eq-3}}{R code example of Equation~ and Equation~}}\label{r-code-example-of-eq-1-and-eq-3}}

In this example, we consider the data generation method with

\begin{itemize}
\tightlist
\item
  \(n=100, p = 1000, m = 5, \zeta = \log(n), {\boldsymbol \Psi}= 0.5{\bf I}_m, \nu = 0.5\).
\end{itemize}

\begin{Shaded}
\begin{Highlighting}[]
\NormalTok{n }\OtherTok{\textless{}{-}} \DecValTok{100}
\NormalTok{p }\OtherTok{\textless{}{-}} \DecValTok{1000}
\NormalTok{m }\OtherTok{\textless{}{-}} \DecValTok{5}
\NormalTok{zeta }\OtherTok{\textless{}{-}} \FunctionTok{log}\NormalTok{(n)}
\NormalTok{Psi }\OtherTok{\textless{}{-}} \FunctionTok{diag}\NormalTok{(}\FloatTok{0.5}\NormalTok{, m)  }\CommentTok{\# Psi}
\NormalTok{v }\OtherTok{\textless{}{-}} \FloatTok{0.5}  \CommentTok{\# nu}
\end{Highlighting}
\end{Shaded}

\begin{itemize}
\tightlist
\item
  and generate data \({\bf Y}={\bf X}{\bf C}+ {\bf E}\), where
  \({\bf E}\sim \mathcal{MN}(0,\bf I_n, {\boldsymbol\Omega})\) with
  \({\boldsymbol\Omega}= 0.2^{|i-j|}\) and
  \({\bf X}\sim\mathcal{N}{\bf 0}, {\boldsymbol \Sigma})\) with
  \({\boldsymbol \Sigma}= 0.2^{|i-j|}\). The true model is
  \({\boldsymbol \gamma}^* = (1,2,3,4,7,8,9,10)\) with their
  coefficients \(\bf C_{\boldsymbol \gamma}\) randomly and independently
  sampled from Uniform\{-1.0, -0.8, -0.6, -0.4, -0.2, 0.0, 0.2, 0.4,
  0.6, 0.8 , 1.0\}. The current model is
  \(\hat{\boldsymbol \gamma}= (1,2,3,4,7,8,9)\) with model size
  \(|\hat{\boldsymbol \gamma}| = 7\).
\end{itemize}

\begin{Shaded}
\begin{Highlighting}[]
\CommentTok{\# Generate data}
\FunctionTok{library}\NormalTok{(mvtnorm)}
\FunctionTok{set.seed}\NormalTok{(}\DecValTok{1314}\NormalTok{)}
\NormalTok{true.model }\OtherTok{\textless{}{-}} \FunctionTok{c}\NormalTok{(}\DecValTok{1}\SpecialCharTok{:}\DecValTok{4}\NormalTok{, }\DecValTok{7}\SpecialCharTok{:}\DecValTok{10}\NormalTok{)  }\CommentTok{\# true model}
\NormalTok{r }\OtherTok{\textless{}{-}} \FunctionTok{c}\NormalTok{(}\DecValTok{1}\SpecialCharTok{:}\DecValTok{4}\NormalTok{, }\DecValTok{7}\SpecialCharTok{:}\DecValTok{9}\NormalTok{)  }\CommentTok{\# current model}
\NormalTok{k }\OtherTok{\textless{}{-}} \FunctionTok{length}\NormalTok{(r)  }\CommentTok{\# current model size}
\NormalTok{rho\_e }\OtherTok{\textless{}{-}} \FloatTok{0.2}
\NormalTok{Omega }\OtherTok{\textless{}{-}}\NormalTok{ rho\_e}\SpecialCharTok{\^{}}\NormalTok{(}\FunctionTok{abs}\NormalTok{(}\FunctionTok{matrix}\NormalTok{(}\DecValTok{1}\SpecialCharTok{:}\NormalTok{m, m, m) }\SpecialCharTok{{-}} \FunctionTok{t}\NormalTok{(}\FunctionTok{matrix}\NormalTok{(}\DecValTok{1}\SpecialCharTok{:}\NormalTok{m, m, m))))}
\NormalTok{rho\_x }\OtherTok{\textless{}{-}} \FloatTok{0.2}
\NormalTok{Sig\_x }\OtherTok{\textless{}{-}}\NormalTok{ rho\_x}\SpecialCharTok{\^{}}\NormalTok{(}\FunctionTok{abs}\NormalTok{(}\FunctionTok{matrix}\NormalTok{(}\DecValTok{1}\SpecialCharTok{:}\NormalTok{p, p, p) }\SpecialCharTok{{-}} \FunctionTok{t}\NormalTok{(}\FunctionTok{matrix}\NormalTok{(}\DecValTok{1}\SpecialCharTok{:}\NormalTok{p, p, p))))}
\NormalTok{seq.p }\OtherTok{\textless{}{-}} \FunctionTok{c}\NormalTok{(}\DecValTok{1}\SpecialCharTok{:}\NormalTok{p)}
\NormalTok{len.true.model }\OtherTok{\textless{}{-}} \FunctionTok{length}\NormalTok{(true.model)}
\CommentTok{\# generate random coefficient matrix C}
\NormalTok{c0 }\OtherTok{\textless{}{-}} \FunctionTok{sample}\NormalTok{(}\FunctionTok{seq}\NormalTok{(}\SpecialCharTok{{-}}\DecValTok{1}\NormalTok{, }\DecValTok{1}\NormalTok{, }\FloatTok{0.2}\NormalTok{), }\AttributeTok{size =}\NormalTok{ len.true.model }\SpecialCharTok{*}\NormalTok{ m, }\AttributeTok{replace =} \ConstantTok{TRUE}\NormalTok{)}
\NormalTok{C }\OtherTok{\textless{}{-}} \FunctionTok{matrix}\NormalTok{(}\DecValTok{0}\NormalTok{, p, m) }\CommentTok{\# initialize C with 0 matrix}
\NormalTok{C[true.model, ] }\OtherTok{\textless{}{-}} \FunctionTok{matrix}\NormalTok{(c0, len.true.model, m)}
\NormalTok{X }\OtherTok{\textless{}{-}} \FunctionTok{rmvnorm}\NormalTok{(n, }\AttributeTok{mean =} \FunctionTok{rep}\NormalTok{(}\DecValTok{0}\NormalTok{, p), }\AttributeTok{sigma =}\NormalTok{ Sig\_x, }\AttributeTok{method =} \StringTok{"chol"}\NormalTok{)}
\NormalTok{E }\OtherTok{\textless{}{-}} \FunctionTok{rmvnorm}\NormalTok{(n, }\AttributeTok{mean =} \FunctionTok{rep}\NormalTok{(}\DecValTok{0}\NormalTok{, m), }\AttributeTok{sigma =}\NormalTok{ Omega, }\AttributeTok{method =} \StringTok{"chol"}\NormalTok{)}
\NormalTok{Y }\OtherTok{\textless{}{-}} \FunctionTok{as.numeric}\NormalTok{(X }\SpecialCharTok{\%*\%}\NormalTok{ C) }\SpecialCharTok{+}\NormalTok{ E}
\end{Highlighting}
\end{Shaded}

To better understand R code and corresponding notations, we list a
cross-reference table for some of them as follows:

\begin{longtable}[]{@{}
  >{\centering\arraybackslash}p{(\columnwidth - 10\tabcolsep) * \real{0.1667}}
  >{\centering\arraybackslash}p{(\columnwidth - 10\tabcolsep) * \real{0.1667}}
  >{\centering\arraybackslash}p{(\columnwidth - 10\tabcolsep) * \real{0.1667}}
  >{\centering\arraybackslash}p{(\columnwidth - 10\tabcolsep) * \real{0.1667}}
  >{\centering\arraybackslash}p{(\columnwidth - 10\tabcolsep) * \real{0.1667}}
  >{\centering\arraybackslash}p{(\columnwidth - 10\tabcolsep) * \real{0.1667}}@{}}
\toprule()
\endhead
I\_n & I\_k1 & log.s.plus1 or log.s.plus2 & rUi & X.rUi & H.rUi \\
\({\bf I}_n\) & \({\bf I}_{k+1}\) &
\(\log({\bf s}({\bf Y}|{\text{nbd}}_+(\hat{\boldsymbol \gamma})))\) &
\(\hat{\boldsymbol \gamma}\cup i\) &
\({\bf X}_{\hat{\boldsymbol \gamma}\cup i}\) &
\({\bf H}_{\hat{\boldsymbol \gamma}\cup i}\) \\
log.s.Y.rUi & I\_k & X.r & X\_r & H.r & colSums(H.r\%*\%X\_r*X\_r) \\
\(\log(s({\bf Y}|\hat{\boldsymbol \gamma}\cup i))\) & \({\bf I}_k\) &
\({\bf X}_{\hat{\boldsymbol \gamma}}\) &
\({\bf X}_{-\hat{\boldsymbol \gamma}}\) &
\({\bf H}_{\hat{\boldsymbol \gamma}}\) &
\({\rm diag}({\bf X}_{-\hat{\boldsymbol \gamma}}^{{ \mathrm{\scriptscriptstyle T} }}{\bf H}_{\hat{{\boldsymbol \gamma}}}{\bf X}_{-\hat{\boldsymbol \gamma}})\) \\
YHX\_r & & & & & \\
\({\bf Y}^{{ \mathrm{\scriptscriptstyle T} }}{\bf H}_{\hat{{\boldsymbol \gamma}}}{\bf X}_{-\hat{\boldsymbol \gamma}}\)
& & & & & \\
\bottomrule()
\end{longtable}

\begin{Shaded}
\begin{Highlighting}[]
\CommentTok{\# For loop method}
\NormalTok{I\_n }\OtherTok{\textless{}{-}} \FunctionTok{diag}\NormalTok{(}\DecValTok{1}\NormalTok{, n)  }\CommentTok{\# n{-}dimension identity matrix }
\NormalTok{I\_k1 }\OtherTok{\textless{}{-}} \FunctionTok{diag}\NormalTok{(}\DecValTok{1}\NormalTok{, k }\SpecialCharTok{+} \DecValTok{1}\NormalTok{) }\CommentTok{\# (k+1){-}dimension identity matrix}
\NormalTok{p\_r }\OtherTok{\textless{}{-}} \FunctionTok{setdiff}\NormalTok{(}\FunctionTok{seq}\NormalTok{(}\DecValTok{1}\NormalTok{, p), r)  }\CommentTok{\# p{-}k vector}
\NormalTok{log.s.plus1 }\OtherTok{\textless{}{-}} \FunctionTok{rep}\NormalTok{(}\ConstantTok{NA}\NormalTok{, }\FunctionTok{length}\NormalTok{(p\_r))}
\NormalTok{j }\OtherTok{\textless{}{-}} \DecValTok{1}
\ControlFlowTok{for}\NormalTok{ (i }\ControlFlowTok{in}\NormalTok{ p\_r) \{}
\NormalTok{  rUi }\OtherTok{\textless{}{-}} \FunctionTok{sort}\NormalTok{(}\FunctionTok{c}\NormalTok{(r, i))  }\CommentTok{\# a model in addition neighbor}
\NormalTok{  X.rUi }\OtherTok{\textless{}{-}}\NormalTok{ X[, rUi]  }\CommentTok{\# n by k+1 submatrix of X}
\NormalTok{  XtX }\OtherTok{\textless{}{-}} \FunctionTok{crossprod}\NormalTok{(X.rUi) }\SpecialCharTok{+} \DecValTok{1}\SpecialCharTok{/}\NormalTok{zeta }\SpecialCharTok{*}\NormalTok{ I\_k1}
\NormalTok{  H.rUi }\OtherTok{\textless{}{-}}\NormalTok{ I\_n }\SpecialCharTok{{-}}\NormalTok{ X.rUi }\SpecialCharTok{\%*\%} \FunctionTok{solve}\NormalTok{(XtX) }\SpecialCharTok{\%*\%} \FunctionTok{t}\NormalTok{(X.rUi)}
  \CommentTok{\# logarithm of Eq (1.1) for a model in additional neighbor}
\NormalTok{  log.s.Y.rUi }\OtherTok{\textless{}{-}} \SpecialCharTok{{-}}\NormalTok{m }\SpecialCharTok{*}\NormalTok{ (k }\SpecialCharTok{+} \DecValTok{1}\NormalTok{)}\SpecialCharTok{/}\DecValTok{2} \SpecialCharTok{*} \FunctionTok{log}\NormalTok{(zeta) }\SpecialCharTok{{-}}\NormalTok{ m}\SpecialCharTok{/}\DecValTok{2} \SpecialCharTok{*} \FunctionTok{log}\NormalTok{(}\FunctionTok{det}\NormalTok{(XtX)) }\SpecialCharTok{{-}}\NormalTok{ (n }\SpecialCharTok{+}\NormalTok{ v)}\SpecialCharTok{/}\DecValTok{2} \SpecialCharTok{*} \FunctionTok{log}\NormalTok{(}\FunctionTok{det}\NormalTok{(}\FunctionTok{t}\NormalTok{(Y) }\SpecialCharTok{\%*\%}\NormalTok{ H.rUi }\SpecialCharTok{\%*\%}\NormalTok{ Y }\SpecialCharTok{+}\NormalTok{ Psi))}
\NormalTok{  log.s.plus1[j] }\OtherTok{\textless{}{-}}\NormalTok{ log.s.Y.rUi}
\NormalTok{  j }\OtherTok{\textless{}{-}}\NormalTok{ j }\SpecialCharTok{+} \DecValTok{1}
\NormalTok{\}}

\CommentTok{\# Proposed Method}
\NormalTok{I\_k }\OtherTok{\textless{}{-}} \FunctionTok{diag}\NormalTok{(}\DecValTok{1}\NormalTok{, k)  }\CommentTok{\# k{-}dimension identity matrix }
\NormalTok{X.r }\OtherTok{\textless{}{-}}\NormalTok{ X[, r] }\CommentTok{\# n by k submatrix of X}
\NormalTok{X\_r }\OtherTok{\textless{}{-}}\NormalTok{ X[, p\_r]  }\CommentTok{\# n by p{-}k m sub{-}matrix of X}
\NormalTok{H.r }\OtherTok{\textless{}{-}}\NormalTok{ I\_n }\SpecialCharTok{{-}}\NormalTok{ X.r }\SpecialCharTok{\%*\%} \FunctionTok{solve}\NormalTok{(}\FunctionTok{crossprod}\NormalTok{(X.r) }\SpecialCharTok{+} \DecValTok{1}\SpecialCharTok{/}\NormalTok{zeta }\SpecialCharTok{*}\NormalTok{ I\_k) }\SpecialCharTok{\%*\%} \FunctionTok{t}\NormalTok{(X.r)  }\CommentTok{\# n by n matrix}
\NormalTok{d }\OtherTok{\textless{}{-}} \DecValTok{1}\SpecialCharTok{/}\NormalTok{zeta }\SpecialCharTok{+} \FunctionTok{colSums}\NormalTok{(H.r }\SpecialCharTok{\%*\%}\NormalTok{ X\_r }\SpecialCharTok{*}\NormalTok{ X\_r)  }\CommentTok{\# p{-}k dimension vector}
\NormalTok{YHX\_r }\OtherTok{\textless{}{-}} \FunctionTok{t}\NormalTok{(Y) }\SpecialCharTok{\%*\%}\NormalTok{ H.r }\SpecialCharTok{\%*\%}\NormalTok{ X\_r  }\CommentTok{\# p{-}k by m matrix}
\NormalTok{YHY\_1 }\OtherTok{\textless{}{-}} \FunctionTok{solve}\NormalTok{(}\FunctionTok{t}\NormalTok{(Y) }\SpecialCharTok{\%*\%}\NormalTok{ H.r }\SpecialCharTok{\%*\%}\NormalTok{ Y }\SpecialCharTok{+}\NormalTok{ Psi)  }\CommentTok{\# m by m matrix}
\NormalTok{u }\OtherTok{\textless{}{-}} \FunctionTok{colSums}\NormalTok{(YHY\_1 }\SpecialCharTok{\%*\%}\NormalTok{ YHX\_r }\SpecialCharTok{*}\NormalTok{ YHX\_r)  }\CommentTok{\# p{-}k dimension vector}
\CommentTok{\# last two items of logarithm of Eq (1.3)}
\NormalTok{log.s.plus1.prop }\OtherTok{\textless{}{-}} \SpecialCharTok{{-}}\NormalTok{m}\SpecialCharTok{/}\DecValTok{2} \SpecialCharTok{*} \FunctionTok{log}\NormalTok{(d) }\SpecialCharTok{{-}}\NormalTok{ (n }\SpecialCharTok{+}\NormalTok{ v)}\SpecialCharTok{/}\DecValTok{2} \SpecialCharTok{*} \FunctionTok{log}\NormalTok{(}\DecValTok{1} \SpecialCharTok{{-}}\NormalTok{ u}\SpecialCharTok{/}\NormalTok{d)}
\CommentTok{\# log(c)}
\NormalTok{log.c }\OtherTok{\textless{}{-}} \SpecialCharTok{{-}}\FloatTok{0.5} \SpecialCharTok{*}\NormalTok{ m }\SpecialCharTok{*}\NormalTok{ (k }\SpecialCharTok{+} \DecValTok{1}\NormalTok{) }\SpecialCharTok{*} \FunctionTok{log}\NormalTok{(zeta) }\SpecialCharTok{{-}} \FloatTok{0.5} \SpecialCharTok{*}\NormalTok{ m }\SpecialCharTok{*} \FunctionTok{log}\NormalTok{(}\FunctionTok{det}\NormalTok{(}\FunctionTok{crossprod}\NormalTok{(X.r) }\SpecialCharTok{+} 
    \DecValTok{1}\SpecialCharTok{/}\NormalTok{zeta }\SpecialCharTok{*}\NormalTok{ I\_k)) }\SpecialCharTok{{-}}\NormalTok{ (n }\SpecialCharTok{+}\NormalTok{ v)}\SpecialCharTok{/}\DecValTok{2} \SpecialCharTok{*} \FunctionTok{log}\NormalTok{(}\FunctionTok{det}\NormalTok{(}\FunctionTok{t}\NormalTok{(Y) }\SpecialCharTok{\%*\%}\NormalTok{ H.r }\SpecialCharTok{\%*\%}\NormalTok{ Y }\SpecialCharTok{+}\NormalTok{ Psi))}
\NormalTok{log.s.plus2 }\OtherTok{\textless{}{-}}\NormalTok{ log.c }\SpecialCharTok{+}\NormalTok{ log.s.plus1.prop  }\CommentTok{\# logarithm of Eq (1.3)}
\end{Highlighting}
\end{Shaded}

There are \(993\) marginal likelihoods in addition neighbor of model
\(\{1,2,3,4,7,8,9\}\) in \texttt{log.s.plus1} and \texttt{log.s.plus2}.
We compute mean absolute percentage error
\(\text{MAPE} = \frac{1}{n}\Sigma_{t=1}^n|\frac{A_t-F_t}{A_t}|\) to
measure the accuracy of the fast computing algorithm.

\begin{Shaded}
\begin{Highlighting}[]
\CommentTok{\# Mean absolute percentage error}
\NormalTok{MAPE }\OtherTok{\textless{}{-}} \FunctionTok{mean}\NormalTok{(}\FunctionTok{abs}\NormalTok{(log.s.plus1 }\SpecialCharTok{{-}}\NormalTok{ log.s.plus2)}\SpecialCharTok{/}\FunctionTok{abs}\NormalTok{(log.s.plus1))}
\FunctionTok{print}\NormalTok{(}\FunctionTok{paste}\NormalTok{(}\StringTok{"MAPE ="}\NormalTok{, MAPE))}
\end{Highlighting}
\end{Shaded}

\begin{verbatim}
[1] "MAPE = 1.55883482799341e-16"
\end{verbatim}

\begin{Shaded}
\begin{Highlighting}[]
\FunctionTok{plot}\NormalTok{(log.s.plus1, log.s.plus2)}
\FunctionTok{abline}\NormalTok{(}\AttributeTok{a =} \DecValTok{0}\NormalTok{, }\AttributeTok{b =} \DecValTok{1}\NormalTok{)}
\end{Highlighting}
\end{Shaded}

\begin{figure}[H]

{\centering \includegraphics{./index_files/figure-pdf/unnamed-chunk-4-1.pdf}

}

\end{figure}

From the plot and MAPE,
\(\log({\bf s}({\bf Y}|{\text{nbd}}_+(\hat{\boldsymbol \gamma})))\)
computed by Equation~\ref{eq-1} and Equation~\ref{eq-3} are the same,
which confirms R code is correct. Next is to compute the time cost.

\bookmarksetup{startatroot}

\hypertarget{simulation-study}{%
\chapter{Simulation study}\label{simulation-study}}

Note that \(\log(c_{\hat{\boldsymbol \gamma}}^+)\) is a constant with
respect to \(i\notin \hat{\boldsymbol \gamma}\) in Equation~\ref{eq-3},
we can ignore it and only compute \texttt{log.s.plus1.prop} to save time
for the purpose of variable selection. We use R package
\texttt{microbenchmark} to conduct the simulation study with the default
replication 100 times.

\begin{Shaded}
\begin{Highlighting}[]
\FunctionTok{library}\NormalTok{(microbenchmark)}
\NormalTok{timecost }\OtherTok{\textless{}{-}} \FunctionTok{microbenchmark}\NormalTok{(}\StringTok{"for\_loop"} \OtherTok{=}\NormalTok{ \{}
\NormalTok{  log.s.plus1 }\OtherTok{\textless{}{-}} \FunctionTok{rep}\NormalTok{(}\ConstantTok{NA}\NormalTok{, }\FunctionTok{length}\NormalTok{(p\_r))}
\NormalTok{  j }\OtherTok{\textless{}{-}} \DecValTok{1}
  \ControlFlowTok{for}\NormalTok{ (i }\ControlFlowTok{in}\NormalTok{ p\_r) \{}
\NormalTok{    rUi }\OtherTok{\textless{}{-}} \FunctionTok{sort}\NormalTok{(}\FunctionTok{c}\NormalTok{(r, i))  }\CommentTok{\# a model in addition neighbor}
\NormalTok{    X.rUi }\OtherTok{\textless{}{-}}\NormalTok{ X[, rUi]  }\CommentTok{\# n by k+1 submatrix of X}
\NormalTok{    XtX }\OtherTok{\textless{}{-}} \FunctionTok{crossprod}\NormalTok{(X.rUi) }\SpecialCharTok{+} \DecValTok{1}\SpecialCharTok{/}\NormalTok{zeta }\SpecialCharTok{*}\NormalTok{ I\_k1}
\NormalTok{    H.rUi }\OtherTok{\textless{}{-}}\NormalTok{ I\_n }\SpecialCharTok{{-}}\NormalTok{ X.rUi }\SpecialCharTok{\%*\%} \FunctionTok{solve}\NormalTok{(XtX) }\SpecialCharTok{\%*\%} \FunctionTok{t}\NormalTok{(X.rUi)}
    \CommentTok{\# logarithm of Eq (1.1) for a model in additional neighbor}
\NormalTok{    log.s.Y.rUi }\OtherTok{\textless{}{-}} \SpecialCharTok{{-}}\NormalTok{m }\SpecialCharTok{*}\NormalTok{ (k }\SpecialCharTok{+} \DecValTok{1}\NormalTok{)}\SpecialCharTok{/}\DecValTok{2} \SpecialCharTok{*} \FunctionTok{log}\NormalTok{(zeta) }\SpecialCharTok{{-}}\NormalTok{ m}\SpecialCharTok{/}\DecValTok{2} \SpecialCharTok{*} \FunctionTok{log}\NormalTok{(}\FunctionTok{det}\NormalTok{(XtX)) }\SpecialCharTok{{-}}\NormalTok{ (n }\SpecialCharTok{+}\NormalTok{ v)}\SpecialCharTok{/}\DecValTok{2} \SpecialCharTok{*} \FunctionTok{log}\NormalTok{(}\FunctionTok{det}\NormalTok{(}\FunctionTok{t}\NormalTok{(Y) }\SpecialCharTok{\%*\%}\NormalTok{ H.rUi }\SpecialCharTok{\%*\%}\NormalTok{ Y }\SpecialCharTok{+}\NormalTok{ Psi))}
\NormalTok{    log.s.plus1[j] }\OtherTok{\textless{}{-}}\NormalTok{ log.s.Y.rUi}
\NormalTok{    j }\OtherTok{\textless{}{-}}\NormalTok{ j }\SpecialCharTok{+} \DecValTok{1}
\NormalTok{  \}}
\NormalTok{\},}
\StringTok{"Proposed"} \OtherTok{=}\NormalTok{ \{}
\NormalTok{I\_k }\OtherTok{\textless{}{-}} \FunctionTok{diag}\NormalTok{(}\DecValTok{1}\NormalTok{, k)  }\CommentTok{\# k{-}dimension identity matrix }
\NormalTok{X.r }\OtherTok{\textless{}{-}}\NormalTok{ X[, r] }\CommentTok{\# n by k submatrix of X}
\NormalTok{X\_r }\OtherTok{\textless{}{-}}\NormalTok{ X[, p\_r]  }\CommentTok{\# n by p{-}k m sub{-}matrix of X}
\NormalTok{H.r }\OtherTok{\textless{}{-}}\NormalTok{ I\_n }\SpecialCharTok{{-}}\NormalTok{ X.r }\SpecialCharTok{\%*\%} \FunctionTok{solve}\NormalTok{(}\FunctionTok{crossprod}\NormalTok{(X.r) }\SpecialCharTok{+} \DecValTok{1}\SpecialCharTok{/}\NormalTok{zeta }\SpecialCharTok{*}\NormalTok{ I\_k) }\SpecialCharTok{\%*\%} \FunctionTok{t}\NormalTok{(X.r)  }\CommentTok{\# n by n matrix}
\NormalTok{d }\OtherTok{\textless{}{-}} \DecValTok{1}\SpecialCharTok{/}\NormalTok{zeta }\SpecialCharTok{+} \FunctionTok{colSums}\NormalTok{(H.r }\SpecialCharTok{\%*\%}\NormalTok{ X\_r }\SpecialCharTok{*}\NormalTok{ X\_r)  }\CommentTok{\# p{-}k dimension vector}
\NormalTok{YHX\_r }\OtherTok{\textless{}{-}} \FunctionTok{t}\NormalTok{(Y) }\SpecialCharTok{\%*\%}\NormalTok{ H.r }\SpecialCharTok{\%*\%}\NormalTok{ X\_r  }\CommentTok{\# p{-}k by m matrix}
\NormalTok{YHY\_1 }\OtherTok{\textless{}{-}} \FunctionTok{solve}\NormalTok{(}\FunctionTok{t}\NormalTok{(Y) }\SpecialCharTok{\%*\%}\NormalTok{ H.r }\SpecialCharTok{\%*\%}\NormalTok{ Y }\SpecialCharTok{+}\NormalTok{ Psi)  }\CommentTok{\# m by m matrix}
\NormalTok{u }\OtherTok{\textless{}{-}} \FunctionTok{colSums}\NormalTok{(YHY\_1 }\SpecialCharTok{\%*\%}\NormalTok{ YHX\_r }\SpecialCharTok{*}\NormalTok{ YHX\_r)  }\CommentTok{\# p{-}k dimension vector}
\CommentTok{\# last two items of logarithm of Eq (1.3)}
\NormalTok{log.s.plus1.prop }\OtherTok{\textless{}{-}} \SpecialCharTok{{-}}\NormalTok{m}\SpecialCharTok{/}\DecValTok{2} \SpecialCharTok{*} \FunctionTok{log}\NormalTok{(d) }\SpecialCharTok{{-}}\NormalTok{ (n }\SpecialCharTok{+}\NormalTok{ v)}\SpecialCharTok{/}\DecValTok{2} \SpecialCharTok{*} \FunctionTok{log}\NormalTok{(}\DecValTok{1} \SpecialCharTok{{-}}\NormalTok{ u}\SpecialCharTok{/}\NormalTok{d)}
\NormalTok{\}}
\NormalTok{) }
\NormalTok{timecost}
\end{Highlighting}
\end{Shaded}

\begin{verbatim}
Unit: milliseconds
     expr        min         lq      mean     median         uq       max neval
 for_loop 144.006329 157.604586 195.91860 169.235529 204.147722 502.21531   100
 Proposed   1.272944   1.322159   2.08821   1.423167   1.652661  24.11825   100
\end{verbatim}

Looking at the median of time cost, the fast computing algorithm is
about 120 times faster than the for-loop method.


\backmatter

\end{document}
